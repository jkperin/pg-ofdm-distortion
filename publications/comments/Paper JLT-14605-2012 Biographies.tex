%% Comments on ""
%% V1.3
%% 30/07/2012
%% by José Paulo

\documentclass[journal]{IEEEtran}
\makeatletter
\def\markboth#1#2{\def\leftmark{\@IEEEcompsoconly{\sffamily}\MakeUppercase{\protect#1}}%
\def\rightmark{\@IEEEcompsoconly{\sffamily}\MakeUppercase{\protect#2}}}
\makeatother
\usepackage[utf8x]{inputenc}
\usepackage[english]{babel}
\usepackage{graphicx}
\usepackage{ucs}
\usepackage{amsmath}
\usepackage{amsfonts}
\usepackage{amssymb}
\usepackage[section]{placeins}
\usepackage{multirow}
\DeclareGraphicsExtensions{.pdf,.png,.jpg}

\hyphenation{op-tical net-works semi-conduc-tor}
\begin{document}

\begin{IEEEbiographynophoto}{José Paulo Krause Perin}
is currently a senior student of the Bachelor of Electrical Engineering program at the Universidade Federal do Espirito Santo, Brazil. His research interests include fiber-optic communication systems and networks.
\end{IEEEbiographynophoto}
\begin{IEEEbiographynophoto}{Moisés Renato Nunes Ribeiro}
was born in Vitoria/ES, Brazil in 1969. He received the B.Sc. degree in electrical engineering from the Instituto Nacional de Telecomunicationes (INATEL), the M.Sc. degree in telecommunications from the Universidade Estadual de Campinas (UNICAMP), Brazil, and the Ph.D. degree in the same area from the University
of Essex, Essex, U.K., in 1992, 1996, and 2002, respectively.
He joined the Department of Electrical Engineering, Universidade Federal do Espirito Santo, Brazil, in 1995. He was a Visiting Professor to the Electrical Engineering Department at Stanford University (October 2010 to September 2011) under a grant from the Brazilian government (CAPES). His research interests include fiber-optic and computer communication devices, systems, and networks.
\end{IEEEbiographynophoto}
\begin{IEEEbiographynophoto}{Adolfo Cartaxo}
(S’89–A’89–M’98–SM’02) was born in Montemor-o-Novo, Portugal,in 1962. He received the ``Licenciatura'' degree in electrical engineering, the M.Sc. degree in telecommunications and computers, and the Ph.D. degree in electrical engineering from the Instituto Superior Técnico (IST), Lisbon Technical University, Lisbon, Portugal, in 1985, 1989, and 1992, respectively. In 1985, he joined the Department of Electrical and Computer Engineering of IST. In 1992, he became an Assistant Professor and he was promoted to Associate Professor in January 2002. Over those years, he has lectured several courses on Telecommunications. He joined the Optical Communications Group of Lisbon Pole of Instituto de Telecomunicações (IT) as a researcher in 1993. He is now a senior researcher conducting research on optical communication systems and networks. Since January 2002, he is member of the National Coordination Commission on Optical Communications of IT. He has been leader of the IST participation or of the Lisbon site of IT in six projects of the European Union programs on R$\&$D in the optical communications area. He has been leader of several national projects in the optical communications area. He is the leader of IST-IT participation in the cooperation project with Brazil in the area of optical networks. He has acted as a technical auditor and evaluator for projects included in ``Advanced Communications Technologies and Services: European RTD'' (ACTS) and ``Information Society Technologies'' (IST) European Union R$\&$D Programs. He has served as a reviewer for the leading international publications in the area of optical communications and networks. He has authored or co-authored more than 60 journal publications (15 as first author) as well as more than 80 international conference papers. He is co-author of two international patents.
Dr. Cartaxo is a Senior member of the IEEE Laser and Electro-Optics Society. 

\end{IEEEbiographynophoto}


\end{document}


