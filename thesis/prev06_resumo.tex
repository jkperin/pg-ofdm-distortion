%%%%%%%%%%%%%%%%%%%%%%%%%%%%%%%%%%%%%%%%%%%%%%%%%%%%%%%%%%%
%%%% resumo
%%%%%%%%%%%%%%%%%%%%%%%%%%%%%%%%%%%%%%%%%%%%%%%%%%%%%%%%%%%
\begin{resumo}
 \begin{Spacing}{1.5}
Neste trabalho ser� estudada uma teoria de grandes sinais para modelar a propaga��o dispersiva de sinais OFDM (\textit{Orthogonal Frequency Division Multiplexing}) em sistemas �pticos IMDD (\textit{Intensity Modulation - Direct Detection}) empregando lasers modulados diretamente com o objetivo de calcular as distor��es causadas pelo efeito combinado entre \emph{chirp} do laser semicondutor e dispers�o crom�tica. A base da teoria apresentada neste trabalho � derivada de um artigo recente, onde s�o apresentadas equa��es anal�ticas para o c�lculo da corrente detectada em sistemas IMDD com lasers modulados diretamente por 1 e por \emph{N} componentes sinusoidais. A partir da corrente detectada pode-se calcular as distor��es induzidas pelo \emph{chirp} do laser. No entanto, a teoria desenvolvida originalmente n�o foi desenvolvida especificamente para sinais OFDM, o que torna impratic�vel o c�lculo das distor��es induzidas pelo \emph{chirp} em sinais OFDM com centenas de subportadoras. Para contornar esse problema, neste trabalho ser�o propostas simplifica��es a essas equa��es de forma a possibilitar o c�lculo, por�m sem prejudicar a precis�o do modelo.

Neste trabalho tamb�m ser� proposto uma teoria de grandes sinais mais precisa do que a apresentada originalmente. A maior precis�o � conseguida em troca da maior complexidade das equa��es. Como consequ�ncia, a teoria desenvolvida n�o pode ser aplicada em sinais OFDM com algumas dezenas de subportadoras.

Por fim, a teoria proposta � aplicada a sinais OFDM de interesse de pesquisa e comercial, mostrando que as simplifica��es n�o comprometem a precis�o da teoria de grandes sinais utilizada. Para todos os casos analisados, o erro calculado em rela��o aos valores esperados pela teoria foram inferiores a 3 dB.
\end{Spacing}
\end{resumo}
